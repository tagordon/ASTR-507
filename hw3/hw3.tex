\documentclass[a4paper, 11pt]{article}
\usepackage{fullpage, amssymb, amsmath, graphicx} 

\newcommand{\mytitle}{Problem Set 3}

\begin{document}
\noindent
\large\textbf{\mytitle} \\ \\ Tyler Gordon \\
\normalsize \today 
\ \ \hrulefill
\section*{Problem 1}
	\subsection*{Part A}
		The entropy is found from
		\begin{equation*}
			S = (U + PV - N\mu)/T
		\end{equation*}
		The pressure and number terms are given in the notes:
		\begin{equation*}
			\frac{PV}{kT} = \frac{4(2s+1)V}{3\pi^{1/2}\lambda^3}F_{3/2}(z)
		\end{equation*}
		\begin{equation*}
			\frac{N\mu}{kT} = \frac{2(2s+1)V}{\pi^{1/2}\lambda^3}F_{1/2}(z)
		\end{equation*}
		with $\lambda=h/\sqrt{2\pi mkT}$
		We can compute the internal energy by integrating the energy of each state 
		weighted by the population in each state:
		\begin{equation*}
			\frac{U}{kT} = \frac{(2s+1)V}{h^3kT}\int_0^\infty \epsilon N(\epsilon) d^3p
		\end{equation*}
		where the factor in front of the integral gives the multiplicity of each energy state. 
		For a non-interacting Fermi gas we can make the substitution $\epsilon=p^2/2m$ and 
		then introduce the variable $w=\epsilon/kT$:
		\begin{equation*}
			\frac{U}{kT} = \frac{4\pi(2s+1)V}{h^3kT}\int_0^\infty N(w) \frac{(2mkT)^{5/2}}{10m}d(w^{5/2})
		\end{equation*}
		where
		\begin{equation*}
			N(w) = \frac{1}{e^w/z + 1}
		\end{equation*}
		$d(w^{5/2}) = (5/2)w^{3/2}dw$, so the integral becomes:
		\begin{equation*}
			\int_0^\infty (mkT)^{5/2}\frac{\sqrt{2}}{m}\frac{w^{3/2}}{e^w/z+1}dw =(mkT)^{5/2}\frac{\sqrt{2}}{m}F_{3/2}(z)
		\end{equation*}
		And therefore the internal energy is:
		\begin{equation*}
			\frac{U}{kT} = \frac{4\pi\sqrt{2}(2s+1)V(mkT)^{3/2}}{h^3}F_{3/2}(z) = \frac{2(2s+1)}{\sqrt{\pi}\lambda^3}F_{3/2}(z)
		\end{equation*}
		where $\lambda = (2\pi mkT)^{3/2}/h^3$. Comparing this to the pressure term, we find that 
		$u = \frac{3}{2}P$. Putting this all together, we find the expression for entropy:
		\begin{equation*}
			S = \frac{5k}{2}\frac{PV}{kT} - \frac{N\mu}{T} = \frac{(2s+1)V}{\sqrt{\pi}\lambda^3}\left(\frac{10}{3}F_{3/2}(z) + \frac{2\mu}{kT}F_{1/2}(z)\right)
		\end{equation*}
	\subsection*{Part B}
		To expand the pressure in a series in fugacity, we can just expand the Fermi-Dirac integral. To do 
		this, we Taylor expand the integrand. The first term in the expansion is just the function evaluated 
		at zero, which is zero. For the linear term we have:
		\begin{equation*}
			\frac{d}{dz}\left(\frac{w^{3/2}}{e^w/z + 1}\right) = \frac{-w^{3/2}}{(e^w/z+1)^2}(-e^w/z^2) = \frac{w^{3/2}e^w}{(e^w + z)^2}
		\end{equation*}
		Evaluating this at $z=0$, $w^{3/2}e^{-w}$. The linear approximation for the Fermi-Dirac integral is 
		then:
		\begin{equation*}
			F_{3/2}(z) \sim z\int_0^\infty w^{3/2}e^{-w}dw
		\end{equation*}
		Integrating this numerically, we have:
		\begin{equation*}
			F_{3/2}(z) \sim 0.47z
		\end{equation*} 
		Adding back the constants to get the full expression for the pressure of the fermi gas:
		\begin{equation*}
			P(z, T) = \frac{1.88z(2s+1)}{3\sqrt{\pi}\lambda^3}
		\end{equation*}
		This indicates that an increase in the degeneracy of the gas (corresponding to an increase in the 
		fugacity) will result in an increase in pressure. 
\section*{Problem 2}
		

\end{document}
